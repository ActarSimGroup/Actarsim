\section{List of messengers in the Geant4 part of ActarSim}\label{sec-messengers}

Next, a complete description of the possible messenger commands\index{messenger commands} available for selecting different options and for the modification of the behavior of the program. The different messenger commands are ordered and separated by their utility, in different directories. The directories are:
\begin{itemize}
\item \textbf{det:} containing the detector commands\index{detector commands}, able to modify the setup comditions. It contains two subdirectories:  det/sil controlling the Silicon Detectors setup and det/sci controlling the Scintillator Detectors setup.
\item \textbf{phys:} containing all related with the interaction physics\index{interaction physics};
\item \textbf{gun:} commands related with the event generation\index{event generation};
\item \textbf{event:} event visualization\index{visualization};
\item \textbf{analControl:} histogramming control\index{histogramming control}.
\end{itemize}

For each case, the command complete name initiates the description, following by the command explanation and some additional features, if required.

\subsection{Commands controlling the detector description:}
\index{detector commands}
\begin{verbatim}
/ActarSim/det/
        Command description.  Just the directory name...

/ActarSim/det/gasGeoIncludedFlag
        Includes the geometry of the gas volume in the simulation (default "off").

/ActarSim/det/silGeoIncludedFlag
        Includes the geometry of the silicons in the simulation (default "off").

/ActarSim/det/sciGeoIncludedFlag
        Includes the geometry of the scintillator in the simulation (default "off").

/ActarSim/det/setMediumMat
        Select Material of the Medium.

/ActarSim/det/setEleField
        Define electric field.
        Usage: /ActarSim/det/setEleField  Ex  Ey  Ez  (in MV/mm)

/ActarSim/det/setMagField
        Define magnetic field.
        Usage: /ActarSim/det/setMagField  Bx  By  Bz  unit

/ActarSim/det/update
        Update geometry.
        This command MUST be applied before \"beamOn\"
        if you changed geometrical value(s) or gas density.

/ActarSim/det/print
        Prints geometry.
\end{verbatim}
 
\subsubsection{gas box detector commands}
 \index{detector commands}
\begin{verbatim}
/ActarSim/det/setGasMat
        Select Material of the Gas.
        DefaultValue("isoC4H10")

/ActarSim/det/gas/setBeamShieldMat
        Select Material of the beam shield.
        DefaultValue("isoC4H10")

/ActarSim/det/gas/setDetectorGeometry
        Select the geometry of the detector.
        DefaultValue("box")

/ActarSim/det/gas/setBeamShield
        Sets a beam shield and selects the geometry.
        DefaultValue("tube")

/ActarSim/det/gas/setBeamShieldMat
        Select Material of the beam shield.
        DefaultValue("isoC4H10")

/ActarSim/det/gas/setXLengthGasBox
        Select the half-length X dimension of the Gas Box

/ActarSim/det/gas/setYLengthGasBox
        Select the half-length Y dimension of the Gas Box

/ActarSim/det/gas/setZLengthGasBox
        Select the half-length Z dimension of the Gas Box

/ActarSim/det/gas/setRadiusGasTub
        Select the external radius of the Gas Tube.

/ActarSim/det/gas/setLengthGasTub
        Select the half-length of the Gas Tube.

/ActarSim/det/gas/setInnerRadiusBeamShieldTub
        Select the external radius of the Gas Tube.

/ActarSim/det/gas/setRadiusBeamShieldTub
        Select the internal radius of the Gas Tube.

/ActarSim/det/gas/setLengthBeamShieldTub
        Select the half-length of the Gas Tube.

/ActarSim/det/gas/print
        Prints geometry.
\end{verbatim}

\subsubsection{silicon detector commands}
\index{detector commands}
\begin{verbatim}
/ActarSim/det/sil/
        Silicon detector control

/ActarSim/det/sil/print
        Prints geometry.

/ActarSim/det/sil/sideCoverage
        Selects the silicon coverage (default 1).
        6 bits to indicate which sci wall is present (1) or absent (0).
        The order is:
        bit1 (lsb) beam output wall
        bit2 lower (gravity based) wall
        bit3 upper (gravity based) wall
        bit4 left (from beam point of view) wall
        bit5 right (from beam point of view) wall
        bit6 (msb) beam entrance wall
        Convert the final binary to a decimal number and set the coverage!

/ActarSim/det/sil/xBoxHalfLength
        Sets the x half length of the silicon detectors box
        DefaultValue(0.5)

/ActarSim/det/sil/yBoxHalfLength
        Sets the y half length of the silicon detectors box
        DefaultValue(0.5)

/ActarSim/det/sil/zBoxHalfLength
        Sets the z half length of the silicon detectors box
        DefaultValue(0.5)
\end{verbatim}

\subsubsection{scintillator detector commands}
\index{detector commands}
\begin{verbatim}
/ActarSim/det/sci/
        Scintillator detector control

/ActarSim/det/sci/print
        Prints geometry.

/ActarSim/det/sci/sideCoverage
        Selects the scintillator coverage (default 1).
        6 bits to indicate which sci wall is present (1) or absent (0).
        The order is:
        bit1 (lsb) beam output wall
        bit2 lower (gravity based) wall
        bit3 upper (gravity based) wall
        bit4 left (from beam point of view) wall
        bit5 right (from beam point of view) wall
        bit6 (msb) beam entrance wall
        Convert the final binary to a decimal number and set the coverage!

/ActarSim/det/sci/xBoxHalfLength
        Sets the x half length of the scintillator detectors box
        DefaultValue(0.5)

/ActarSim/det/sci/yBoxHalfLength
        Sets the y half length of the scintillator detectors box
        DefaultValue(0.5)

/ActarSim/det/sci/zBoxHalfLength
        Sets the z half length of the scintillator detectors box
        DefaultValue(0.5)
\end{verbatim}

\textbf{IMPORTANT NOTE}: \textit{/ActarSim/det/update} should be called after each geometrycal modification. This command MUST be applied before \textit{/run/beamOn}.

Note: the geometries of sil and sci detectors are hard-coded right now, it should be modified to cope with different setups and possibilities.

\subsection{physics processes commands}
\index{interaction physics}
The full bank of particles  (ConstructBosons(); ConstructLeptons(); ConstructMesons(); ConstructBaryons(); G4ShortLivedConstructor and
G4IonConstructor) are constructed. The physics processes (transportation, em and decay) are now taken from the \textit{examples/extended/medical/GammaTherapy} (basically the PhysicsList implementation for em processes). Other hadronic particles are proccesses are also taken from this example. Note that the PhysicsList can be selected between several possibilities:

\begin{verbatim}
             em: standard, lowenergy, penelope, (choose one from these three)
         common: decay
       hadronic: elastic, binary, binary_ion, gamma_nuc
 ion low-energy: ion-LowE, ion-LowE-ziegler1977, ion-LowE-ziegler1985,
                 ion-LowE-ziegler2000, ion-standard
\end{verbatim}

\begin{verbatim}
/ActarSim/phys/
        Physics list commands directory

/ActarSim/phys/setGCut
        Set gamma cut.

/ActarSim/phys/setECut
        Set electron cut.

/ActarSim/phys/setPCut
        Set positron cut.

/ActarSim/phys/setCuts
        Set cut for all.

/ActarSim/phys/addPhysics
        Add modula physics list.

/ActarSim/phys/verbose
        Set verbose level for processes
\end{verbatim}

Note that a \textit{/run/initialization} is now needed to initialize the application. This initialization has been removed from the \textit{ActarSim.cc} main function to permit the definition of different physics (lowenergy, penelope...) from the PhysicsList before the initialization. An example of how to initialize is shown in the macros (test1.mac, for instance).

\subsection{event generation commands}

The event generation\index{event generation} can be controled by user commands under the directory \textit{/ActarSim/gun} and subdirectories inside. There are several types of reactions that the program handles:

\begin{verbatim}
A) Track a particle or set of particles defined from the Particles list. 
This mode is selected setting the commands 
        /ActarSim/gun/reactionFromFile off (default behavior)
        /ActarSim/gun/reactionFromCine off 
        ...
B) Track a predefined reaction from a file 
C) Track a reaction calculated from CINE (program from W. Mittig)
D) Track a reaction calculated from KINE (program from M.S. Golovkov)
\end{verbatim}

For each case, a realistic interaction (beam distribution and interaction over all gas volume, plus the energy lost of the beam in the gas before the reaction). These commands are described below:

\begin{verbatim}
/ActarSim/gun/realisticBeam
        Simulates beam emittance according to emittance parameters.
         Choice : on, off(default)

/ActarSim/gun/beamInteraction
        Simulates the beam energy loss in gas.
         Choice : on, off(default)

/ActarSim/gun/emittance
        Selects the value of the emittance [in mm mrad].  
        Default value is 1 mm mrad. 

/ActarSim/gun/beamRadiusAtEntrance
        Selects the beam radius at entrance of ACTAR.
        Used with the emittance to calculate the position and angle
          distributions of the beam when a realisticBeam option is set.
\end{verbatim}

\subsubsection{Track a particle or set of particles defined from the Particles list}

\begin{verbatim}
/ActarSim/gun/List
        List available particles. Invoke G4ParticleTable.

/ActarSim/gun/particle
        Select the incident particle (proton is default)
        (ion can be specified for shooting ions).

/ActarSim/gun/ion
        Set properties of ion to be generated.
        [usage] /gun/ion Z A Q E
                Z:(int) AtomicNumber
                A:(int) AtomicMass
                Q:(int) Charge of Ion (in unit of e)
                E:(double) Excitation energy (in keV)

/ActarSim/gun/energy
        Sets the kinetic energy of the primary particle
        DefaultValue(20.)

/ActarSim/gun/direction
        Set momentum direction.
        Direction needs not to be a unit vector.

/ActarSim/gun/position
        Set starting position of the particle.

/ActarSim/gun/time      
        Set initial time of the particle.

/ActarSim/gun/polarization
        Set polarization.

/ActarSim/gun/number
        Set number of particles to be generated.

/ActarSim/gun/randomVertexZPosition on
        Set the vertex Z-value random or not
        Choice: on (default), off

/ActarSim/gun/randomVertexZRange min max unit
        This command is effective only if randomVertexZPosition=="on"
        min: minimum vertex Z-value, default: 0
        max: maximum vertex Z-value, default: 300
        unit: defaule mm

/ActarSim/gun/vertexZPosition z-value unit
        This command if effective only if randomVertexZPosition=="off"
        z-value: user-given vertex Z-value
        unit: default mm
\end{verbatim}


\subsubsection{Track a predefined reaction from a file}

\begin{verbatim}
/ActarSim/gun/reactionFromFile
        Select a reaction from an input file
        Choice : on, off(default)

/ActarSim/gun/reactionFile
        Select the reaction definition file
\end{verbatim}


\subsubsection{Track a reaction calculated from CINE (program from W. Mittig)}
\index{CINE}
\begin{verbatim}
/ActarSim/gun/reactionFromCine
        Select a reaction using Cine
        Choice : on, off(default)

Note that the following commands are under the subdirectory Cine

/ActarSim/gun/Cine/randomTheta
        Select a random Theta angle for the scattered particle.
        Choice : on(default), off

/ActarSim/gun/Cine/randomThetaVal
        Sets the limit in the Theta angle for the scattered particle.
        The value is randomly chosen between the limits (degrees)
               [usage] /ActarSim/gun/Cine/incidentIon min max 
                min: the minimum angle (default 0 degrees)
                        max: the maximum angle (default 180 degrees)

/ActarSim/gun/Cine/thetaLabAngle        
        Sets theta lab angle for the scattered particle (degrees)
        DefaultValue(0.5)

/ActarSim/gun/Cine/incidentIon
        Set properties of incident ion to be generated.
        [usage] /ActarSim/gun/Cine/incidentIon Z A Q E
                Z:(int) AtomicNumber
                A:(int) AtomicMass
                Q:(int) Charge of ion (in unit of e)
                E:(double) Excitation energy (in keV)

/ActarSim/gun/Cine/targetIon
        Set properties of target ion to be generated.
        [usage] /ActarSim/gun/Cine/targetIon Z A Q E
                Z:(int) AtomicNumber
                A:(int) AtomicMass
                Q:(int) Charge of ion (in unit of e)
                E:(double) Excitation energy (in keV)

/ActarSim/gun/Cine/scatteredIon
        Set properties of scattered ion to be generated.
        [usage] /ActarSim/gun/Cine/scatteredIon Z A Q E
                Z:(int) AtomicNumber
                A:(int) AtomicMass
                Q:(int) Charge of ion (in unit of e)
                E:(double) Excitation energy (in keV)

/ActarSim/gun/Cine/recoilIon
        Set properties of recoil ion to be generated.
        [usage] /ActarSim/gun/Cine/recoilIon Z A Q E
                Z:(int) AtomicNumber
                A:(int) AtomicMass
                Q:(int) Charge of ion (in unit of e)
                E:(double) Excitation energy (in keV)

/ActarSim/gun/Cine/reactionQ  
        Sets the reaction Q-value for the ground state (MeV)

/ActarSim/gun/Cine/labEnergy
        Sets the laboratory energy (MeV)
\end{verbatim}

\subsubsection{Track a reaction calculated from KINE (program from M.S. Golovkov)}
\index{KINE}
\begin{verbatim}
/ActarSim/gun/reactionFromKine
        Select a reaction using KINE
        Choice : on(default), off

Note that the following commands are under the subdirectory Kine

/ActarSim/gun/Kine/randomTheta
    Select a random Theta angle for the scattered particle.
    Choice : on(default), off

/ActarSim/gun/Kine/randomThetaVal
    Set the range of theta angles if it is to be randomly chosen
    [usage] /ActarSim/gun/Kine/randomThetaVal min, max
            min: the minimum angle (default 0 degrees)
            max: the maximum angle (default 180 degrees)

/ActarSim/gun/Kine/incidentIon
    Set the properties of the incident ion
    [usage] /ActarSim/gun/Kine/incidentIon Z A Q Ex Mass
            Z: atomic number (I)
            A: mass number   (I)
            Q: charge number (I)
           Ex: excitation energy (D, in MeV)
         Mass: mass (D, in u)  

/ActarSim/gun/Kine/targetIon
    Set the properties of the target ion
    [usage] /ActarSim/gun/Kine/targetIon Z A Q Ex Mass
            Z: atomic number (I)
            A: mass number   (I)
            Q: charge number (I)
           Ex: excitation energy (D, in MeV)
         Mass: mass (D, in u)  

/ActarSim/gun/Kine/scatteredIon
    Set the properties of the scattered ion
    [usage] /ActarSim/gun/Kine/scatteredIon Z A Q Ex Mass
            Z: atomic number (I)
            A: mass number   (I)
            Q: charge number (I)
           Ex: excitation energy (D, in MeV)
         Mass: mass (D, in u)  

/ActarSim/gun/Kine/recoilIon
    Set the properties of the recoiled ion
    [usage] /ActarSim/gun/Kine/recoilIon Z A Q Ex Mass
            Z: atomic number (I)
            A: mass number   (I)
            Q: charge number (I)
           Ex: excitation energy (D, in MeV)
         Mass: mass (D, in u)  

/ActarSim/gun/Kine/labEnergy
    Set the incident energy
    [usage] /ActarSim/gun/Kine/labEnergy E MeV
            E: incident energy (D)
          MeV: the unit

/ActarSim/gun/Kine/randomThetaCM
    Randomize or not the Theta_cm value
    Choice: on (default), off
    Default: on

/ActarSim/gun/Kine/randomThetaRange min max unit
    This command is effective only if randomThetaCM=="on"
    min: default value 0.
    max: default value 180.
    unit: default deg

/ActarSim/gun/Kine/randomPhiAngle
    Randomize or not the Phi angle, useful when testing analyzing macros
    Choice: on (default), off
    default: on

/ActarSim/gun/Kine/userThetaCM theta_cm unit
    This command is effective only if randomThetaCM is "off"
    theta_cm: user given Theta_cm for KINE
    unit: default deg

/ActarSim/gun/Kine/userPhiAngle Phi unit
    This command is effective only if randomPhiAngle is "off"
    Phi: user given phi angle of the recoiled particle,
         e.g., proton in the 78Ni(d,p)79Ni reaction in inverse kinematics
    unit: default deg
\end{verbatim}

\subsection{event visualization}

Inside the \textit{ActarSimEventAction} one can chose the visualization\index{visualization} of the tracks in the event, in particular the user command:

\begin{verbatim}
/ActarSim/event/drawTracks
        Draw the tracks in the event (Choice : none, charged, neutral, all(default))
        DefaultValue("all");

/ActarSim/event/printModulo
        Print events (modulo n, that is prints every n event)
\end{verbatim}

\subsection{histogramming control}

The Tree and histograms\index{histogramming control} can be controled by user commands under the directory \textit{/ActarSim/analControl} and subdirectories inside:

\begin{verbatim}
/ActarSim/analControl/storeTracks
        Store the tracks in the output Tree
        Choice : on, off(default)

/ActarSim/analControl/storeTrackHistos
        Store the tracks in Histograms
        Choice : on, off(default)

/ActarSim/analControl/storeEvents
        Store the events in the output Tree
        DefaultValue("off")

/ActarSim/analControl/storeSimpleTracks
        Store the simple tracks in the output Tree
        DefaultValue("off")

/ActarSim/analControl/storeHistograms
        Store the events in the output Tree
        Choice : on, off(default)
\end{verbatim}
